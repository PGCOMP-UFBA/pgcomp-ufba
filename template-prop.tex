%% Template para dissertacaoo/tese na classe UFBAthesis
%% versao 1.0
%% (c) 2005 Paulo G. S. Fonseca
%% (c) 2012 Antonio Terceiro
%% (c) 2014 Christina von Flach
%% www.dcc.ufba.br/~flach/ufbathesis

%% Carrega a classe ufbathesis
%% Opcoes: * Idiomas
%%           pt   - portugues (padrao)
%%           en   - ingles
%%         * Tipo do Texto
%%           bsc  - para monografias de graduacao
%%           msc  - para dissertacoes de mestrado (padrao)
%%           qual - exame de qualificacao de mestrado
%%           prop - exame de qualificacao de doutorado
%%           phd  - para teses de doutorado
%%         * Media
%%           scr  - para versao eletronica (PDF) / consulte o guia do usuario
%%         * Estilo
%%           classic - estilo original a la TAOCP (deprecated) - apesar de deprecated, manter esse.
%%           std     - novo estilo a la CUP (padrao)
%%         * Paginacao
%%           oneside - para impressao em face unica
%%           twoside - para impressao em frente e verso (padrao)

% Atencao: Manter 'classic' na declaracao abaixo:
\documentclass[en, prop, classic, a4paper]{ufbathesis}

%% Preambulo:
\usepackage[T1]{fontenc} %muda a codificação da fonte, tornando possível copiar textos com acentos no arquivo de saída, pdf.
\usepackage{lmodern} %usado com [T1], tornará as fontes com contornos de alta qualidade para as três famílias de fontes LaTeX.
\usepackage[utf8]{inputenc}
\usepackage{graphicx}
\usepackage{lipsum}
\usepackage{hyphenat}
\usepackage[usenames, dvipsnames, table]{xcolor}
\usepackage{booktabs}
\usepackage{pifont}
\usepackage{multirow}
\usepackage{listings}
\usepackage{colortbl}
\usepackage{xfrac}
\usepackage[FIGTOPCAP]{subfigure}
\usepackage[printonlyused, withpage]{acronym}


% Universidade
\university{Universidade Federal da Bahia}

% Endereco (cidade)
\address{Salvador}

% Instituto ou Centro Academico
\institute{Instituto de Matem\'{a}tica}

% Nome da biblioteca - usado na ficha catalografica
\library{Biblioteca Reitor Mac\^{e}do Costa}

% Programa de pos-graduacao
\program{Programa de P\'{o}s-Gradua\c{c}\~{a}o em Ci\^{e}ncia da Computa\c{c}\~{a}o}

% Area de titulacao
\majorfield{Ci\^{e}ncia da Computa\c{c}\~{a}o}

% Titulo da dissertacao ou tese
\title{T\'{\i}tulo da Disserta\c{c}\~{a}o ou Tese}

% Data da defesa
% e.g. \date{19 de fevereiro de 2013}
\date{13 de janeiro de 2014}
% e.g. \defenseyear{2013}
\defenseyear{2014}

% Autor
% e.g. \author{Jose da Silva}
\author{Nome Completo do AUTOR}

% Orientador(a)
% Opcao: [f] - para orientador do sexo feminino
% e.g. \adviser[f]{Profa. Dra. Maria Santos}
\adviser[f]{Nome Completo da ORIENTADORA}

% Orientador(a)
% Opcao: [f] - para orientador do sexo feminino
% e.g. \coadviser{Prof. Dr. Pedro Pedreira}
% Comente se nao ha co-orientador
\coadviser{Nome Completo do CO-ORIENTADOR}

%% Inicio do documento
\begin{document}

\pgcompfrontpage

%% Parte pre-textual
\frontmatter

\pgcomppresentationpage


%%%%%%%%%%%%%%%%%%%%%
% Resumo em Portugues
%%%%%%%%%%%%%%%%%%%%%

\resumo
COLOQUE O RESUMO. Se preferir, crie um arquivo separado e o inclua via comando include.

Para evitar problemas de formato neste template (de uso geral), usamos acentua\c{c}\~{a}o mostrada abaixo. 

\begin{verbatim} 
\c{c} \~{a} \'{a} \^{e} \'{\i}
\end{verbatim} 

N\~{a}o precisa fazer dessa forma, caso use pacotes adequados (latin1, etc.).

% Palavras-chave do resumo em Portugues
\begin{keywords}
PALAVRAS-CHAVE.
\end{keywords}

%%%%%%%%%%%%%%%%%%%
% Resumo em Ingles
%%%%%%%%%%%%%%%%%%%

\abstract
COLOQUE O RESUMO EM INGL\^{E}S. Se preferir, crie um arquivo separado e o inclua via comando include.
% Palavras-chave do resumo em Ingles
\begin{keywords}
PALAVRAS-CHAVE EM INGL\^{E}S.
\end{keywords}

%%%%%%%%%%%%%%%%%%%
% Sumario / Indice
%%%%%%%%%%%%%%%%%%%

% Comente para ocultar
\tableofcontents

% Lista de figuras
% Comente para ocultar
\listoffigures

% Lista de tabelas
% Comente para ocultar
\listoftables

\chapter*{Lista de Siglas}

% Sintaxe da lista de acordo com a documentação do pacote `acronym'
% documentação: http://mirror.unl.edu/ctan/macros/latex/contrib/acronym/acronym.pdf
\begin{acronym}[PGCOMP]
    \acro{PGCOMP}{Programa de Pós-Graduação em Ciência da Computação}
    \acro{CNPq}{Conselho Nacional de Desenvolvimento Científico e Tecnológico}
\end{acronym}

%% Parte textual
\mainmatter

% Eh aconselhavel criar cada capitulo em um arquivo separado, digamos
% "capitulo1.tex", "capitulo2.tex", ... "capituloN.tex" e depois
% inclui-los com:
% \include{capitulo1}
% \include{capitulo2}
% ...
% \include{capituloN}
%
% Importante: 
% Use \xchapter{}{} ao inves de \chapter{}; se n�o quiser colocar texto antes do inicio do capitulo, use \xchapter{texto}{}.

\xchapter{Introdu\c{c}\~{a}o}{Este eh o primeiro cap\'{\i}tulo, onde eu conto toda a historia deste trabalho, o problema, a solu\c{c}\~{a}o, etc.}

% É recomendável utilizar `\acresetall' no início de cada capítulo para reiníciar o contator de referências às siglas.
\acresetall 

\section{Se\c{c}\~{a}o}
Trabalho do  \ac{PGCOMP}. Bolsa do \ac{CNPq}.

\begin{figure}[h]
    Figure
    \caption{As siglas também funcionam nas legendas, seja na forma de sigla \ac{CNPq}, seja na forma completa \acf{PGCOMP}.}
\end{figure}

\lipsum

\subsection{Uma Subse\c{c}\~{a}o}
\acresetall
Texto para mostrar como o \verb|\acresetall| funciona \ac{CNPq}, \ac{PGCOMP}. Ele reseta os contadoes e faz a sigla aparecer na forma estendida novamente.

\subsection{Outra Subse\c{c}\~{a}o}

Texto  \acf{CNPq}, \acf{PGCOMP}.

\xchapter{Revis\~{a}o Bibliogr\'{a}fica}{Neste cap\'{\i}tulo eu apresento todo o material que eu estudei durante a elabora\c{c}\~{a}o do trabalho.}

\lipsum

Livro \cite{demeyer2008} e  livro \cite{raymond1999}.

\xchapter{Exemplos}{} %sem preambulo

A numera\c{c}\~{a}o de figuras \'{e} sequencial, dentro do cap\'{\i}tulo. Ver Figura \ref{default-regular1} e Figura \ref{default-regular2}.

A numera\c{c}\~{a}o de tabelaas \'{e} sequencial, dentro do cap\'{\i}tulo. Ver Tabela \ref{default-table1} e Tabela \ref{default-table2}.


\section{Exemplos de Figura}

\begin{figure}[htbp]
\begin{center}
  \includegraphics[scale=0.5]{ufba.eps}
\caption{Bras\~{a}o da UFBA - Menor.}
\label{default-regular1}
\end{center}
\end{figure}

\begin{figure}[htbp]
\begin{center}
  \includegraphics[scale=0.75]{ufba.eps}
\caption{Bras\~{a}o da UFBA - Maior.}
\label{default-regular2}
\end{center}
\end{figure}

\lipsum

\section{Exemplos de Tabela}
\subsection{Uma Tabela}
\begin{table}[htbp]
\caption{Uma tabela com 3 linhas e 2 colunas.}
\begin{center}
\begin{tabular}{|c|c|} 
\hline
elemento 11 & elemento 12 \\ \hline
elemento 21 & elemento 22 \\ \hline
elemento 31 & elemento 32 \\
\hline
\end{tabular}
\end{center}
\label{default-table1}
\end{table}%

\lipsum

\begin{table}[htbp]
\caption{Uma tabela com 3 linhas e 3 colunas.}
\begin{center}
\begin{tabular}{|l|c|c|} 
\hline
elemento 11 & elemento 12 & elemento 13\\ \hline
elemento 21 & elemento 22 & elemento 23\\ \hline
elemento 31 & elemento 32 & elemento 33\\
\hline
\end{tabular}
\end{center}
\label{default-table2}
\end{table}%

\xchapter{Outro cap\'{\i}tulo}{} %sem preambulo
\lipsum


%% Parte pos-textual
\backmatter

% Bibliografia
% � aconselh�vel utilizar o BibTeX a partir de um arquivo, digamos "biblio.bib".
% Para ajuda na cria��o do arquivo .bib e utiliza��o do BibTeX, recorra ao
% BibTeXpress em www.cin.ufpe.br/~paguso/bibtexpress
\bibliographystyle{abntex2-alf}
\bibliography{biblio}

% Apendices
% Comente se naoo houver apendices
\appendix

\xchapter{Exemplo de Ap\^endice}{} %sem preambulo
\lipsum
% Eh aconselhavel criar cada apendice em um arquivo separado, digamos
% "apendice1.tex", "apendice.tex", ... "apendiceM.tex" e depois
% inclui--los com:
% \include{apendice1}
% \include{apendice2}
% ...
% \include{apendiceM}

%% Fim do documento
\end{document}
%------------------------------------------------------------------------------------------%
