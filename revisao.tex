% Strings de busca 
%
% (((((clustering) OR cluster labeling) OR cluster description) AND fuzzy AND( document OR text mining)))
% http://ieeexplore.ieee.org/search/searchresult.jsp?queryText=(((((clustering)%20OR%20cluster%20labeling)%20OR%20cluster%20description)%20AND%20fuzzy%20%20AND(%20document%20OR%20text%20mining)))&ranges=2010_2016_Year&matchBoolean=true&searchField=Search_All

% ((clustering OR "cluster label*" OR "cluster descriptors") AND fuzzy AND (document OR "text mining" OR "document organization" OR "soft document" OR "text data"))
% http://ieeexplore.ieee.org/search/searchresult.jsp?queryText=((clustering%20OR%20.QT.cluster%20label*.QT.%20OR%20.QT.cluster%20descriptors.QT.)%20AND%20fuzzy%20AND%20(document%20OR%20.QT.text%20mining.QT.%20OR%20.QT.document%20organization.QT.%20OR%20.QT.soft%20document.QT.%20OR%20.QT.text%20data.QT.))&sortType=desc_p_Publication_Year&matchBoolean=true&searchField=Search_All

\section{Considerações Iniciais}

A proposta de organização flexível de documentos está relacionada a vários campos de estudo, 
como ficou evidenciado na fundamentação teórica. Por isso a literatura existente para essa proposta 
é bastante rica e densa. Portanto com o propósito de otimizar a atividade de pesquisa e seleção
do conhecimento científico produzido a respeito do tema, foram utilizados algumas técnicas de 
revisão sistemática de literatura ($SLR\ -\ Sistematic\ Literature\ Review$) utilizadas em \cite{Rios2010}. 
Com o objetivo de estabelecer critérios mais precisos na fase inicial da descoberta de conteúdo
científico relacionado ao tema. Foi então adotada uma técnica comum ao método SLR, que consiste na 
elaboração de uma uma string de busca, usando operadores lógicos. Estabelecendo assim uma 
maneira mais objetiva para a obtenção de resultados relevantes a proposta dessa monografia.
Portanto, levando em consideração os tópicos chaves e a proposta desse trabalho, foi 
construída a seguinte string de busca: 

\begin{multline}
(clustering\ OR\ "cluster\ label*"\ OR\ "cluster\ descriptors")\ AND\ fuzzy \\ 
AND\ (document\ OR\ "text\ mining"\ OR\ "document\ organization"\ OR\ \\
"soft\ document"\ OR\ "text\ data")
\label{eq:busca}
\end{multline}

Devido o amplo acervo de publicações científicas presentes no repositório 
IEEExplore\footnote{http://ieeexplore.ieee.org/}, assim como também a possibilidade de se utilizar
operadores lógicos e buscas parametrizadas. Foi realizado então uma busca no repositório 
IEEExplore, restringindo o período de resultados entre os anos de 2010 e 2016, permitindo então que 
os resultados obtidos fossem mais recentes.

Com base nos resultados obtidos, foi realizada a leitura dos títulos e resumos
dos artigos, com o propósito de descartar resultados com baixa relevância para essa pesquisa.
Durante a fase de leitura parcial dos resultados da busca, foram agrupados os artigos em 
três categorias: agrupamento fuzzy, extração de descritores e organização flexível de documentos.
As publicações selecionadas e direcionadas para a categoria de agrupamento fuzzy, foram as que
possuíam propostas de alteração de métodos de agrupamento existentes ou novos métodos. Enquanto
artigos que tinham como conteúdo a análise dos termos de uma coleção, critério de seleção de termos
ou atribuição de termos a grupos de documentos, foram agrupados na categoria de extração de 
descritores. Por fim, artigos mais gerais, propondo métodos ou realizando revisões de métodos,
pertinentes ao processo de organização de documentos textuais, foram categorizados no grupo 
de organização flexível de documentos.

Para complementar os resultados obtidos foram adicionados artigos de alta relevância para o tema,
e que apesar de serem antigos, ainda são amplamente citados em pesquisas recentes. Muitos desses
artigos como é o caso do método FCM proposto em \cite{Bezdek1984}, são pilares fundamentais para
o tema.

Nas próximas seções contém a revisão das pesquisas selecionadas, onde é elucidado os pontos
chaves de cada pesquisa, a definição das propostas contida nos artigos e por fim a conexão com o 
objetivo dessa monografia.

\section{Organização Flexível de Documentos}

Após a proposição da lógica fuzzy que se propunha a lidar com a incerteza e imprecisão em 
\cite{Zadeh1965}, foi possível a elaboração de diversos métodos que se utilizassem dos benefícios
da lógica fuzzy e aplicassem a diversos problemas do mundo real. Este é o caso da organização
de documentos, que por não ser uma tarefa precisa, necessita de uma certa flexibilidade no 
processo.

\cite{Matsumoto10} informa que os mecanismos adotados em sistemas de recuperação de informação 
(SRI), tais como buscadores web, estão dispostos em dois grupos. Sendo que o primeiro tem como
foco o usuário realizando a busca, a qual é comumente chamada de busca web personalizada. 
Nessa abordagem os resultados obtidos são ordenados de acordo com a relevância do resultado
para o usuário. Para calcular essa relevância, os buscadores realizam tarefas de 
coleta de dados dos usuários e comparação das preferências com demais usuários do sistema.  
Enquanto na segunda abordagem os resultados da busca é categorizado em grupos, permitindo assim
que o usuário decida em qual grupo ele pretende visualizar as informações. Por exemplo, quando um usuário pesquisar pelo termo java, os resultados poderiam ser agrupados nas seções: máquina virtual,
linguagem java, programas em java, oracle e etc. Baseada nessa abordagem de categorização 
de resultados em SRIs, \cite{MarcaciniR10} propõe uma abordagem de agrupamento incremental 
e hierárquico para construção dos tópicos dos documentos, a qual permite a atualização das 
categorias a medida que novos documentos são adicionados sem realizar a etapa de agrupamento 
novamente. É possível a visualização dessa abordagem de categorização hierárquica, 
através da ferramenta online 
Torch\footnote{http://sites.labic.icmc.usp.br/torch/webcluster/},
dos autores do artigo.

