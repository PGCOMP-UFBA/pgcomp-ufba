%% Template para dissertação/tese na classe UFBAthesis
%% versao 1.0
%% (c) 2005 Paulo G. S. Fonseca
%% (c) 2012 Antonio Terceiro
%% (c) 2014 Christina von Flach
%% www.dcc.ufba.br/~flach/ufbathesis

%% Carrega a classe ufbathesis
%% Opcoes: * Idiomas
%%           pt   - portugues (padrao)
%%           en   - ingles
%%         * Tipo do Texto
%%           bsc  - para monografias de graduacao
%%           msc  - para dissertacoes de mestrado (padrao)
%%           qual - exame de qualificacao de mestrado
%%           prop - exame de qualificacao de doutorado
%%           phd  - para teses de doutorado
%%         * Mídia
%%           scr  - para versão eletrônica (PDF) / consulte o guia do usuario
%%         * Estilo
%%           classic - estilo original a la TAOCP (deprecated)
%%           std     - novo estilo a la CUP (padrao)
%%         * Paginacao
%%           oneside - para impressao em face unica
%%           twoside - para impressao em frente e verso (padrao)
\documentclass[msc, classic, a4paper]{ufbathesis}

%% Preambulo:
\usepackage[utf8]{inputenc}
\usepackage{graphicx}
\usepackage{lipsum}

% Universidade
\university{UNIVERSIDADE FEDERAL DA BAHIA}

% Endereco (cidade)
\address{Salvador}

% Instituto ou Centro Academico
\institute{INSTITUTO DE MATEM\'{A}TICA}

% Nome da biblioteca - usado na ficha catalografica
\library{BIBLIOTECA REITOR MAC\^{E}DO COSTA}

% Programa de pos-graduacao
\program{PROGRAMA DE P\'{O}S-GRADUA\c{C}\~{A}O EM CI\^{E}NCIA DA COMPUTA\c{C}\~{A}O}

% �rea de titulacao
\majorfield{CI\^{E}NCIA DA COMPUTA\c{C}\~{A}O}

% Titulo da dissertacao
\title{TITULO DA DISSERTACAO}

% Data da defesa
% e.g. \date{19 de fevereiro de 2003}
\date{DATA DA DEFESA}

% Autor
% e.g. \author{Jose da Silva}
\author{NOME DO AUTOR}

% Orientador(a)
% Opcao: [f] - para orientador do sexo feminino
% e.g. \adviser[f]{Profa. Dra. Maria Santos}
\adviser{NOME DO(DA) ORIENTADOR(A)}

% Orientador(a)
% Opcao: [f] - para orientador do sexo feminino
% e.g. \coadviser{Prof. Dr. Pedro Pedreira}
% Comente se nao se aplicar
\coadviser{NOME DO(DA) CO-ORIENTADOR(A)}

%% Inicio do documento
\begin{document}

\pgcompfrontpage{}

%% Parte pre-textual
\frontmatter

\pgcomppresentationpage

% Ficha catalografica
\authorcitationname{SEU NOME EM CITACOES} % e.g. Terceiro, Antonio Soares de Azevedo
\advisercitationname{NOME DO SEU ORIENTADOR EM CITACOES} % e.g. Chavez, Christina von Flach Garcia
\coadvisercitationname{NOME DO SEU CO-ORIENTADOR EM CITACOES} % e.g. Mendonca, Manoel Gomes de
\catalogtype{Disserta\c{c}\~{a}o (Mestrado)} % e.g. ``Tese (Doutorado)''
\catalogtopics{TOPICOS PARA FICHA CATALOGRAFICA} % e.g. ``1. Complexidade Estrutural. 2. Engenharia de Software''
\catalogcdd{NUMERO CDD} % e.g. ``CDD 20.ed. XXX.YY'' (esse número vai lhe ser dado pela biblioteca)
\catalogingsheet

% Termo de aprovacaoo
\approvalsheet{Salvador, DIA de MES de ANO}{
   \comittemember{Profa. Dra. Professora 1}{Universidade XYZ}
   \comittemember{Prof. Dr. Professor 2}{Universidade 123}
   \comittemember{Profa. Dra. Professora 3}{Universidade ABC}
}

% Dedicatoria
% Comente para ocultar
\begin{dedicatory}
Coloque sua DEDICATORIA AQUI.
\end{dedicatory}

% Agradecimentos
\acknowledgements
Coloque seus AGRADECIMENTOS AQUI.

% Epigrafe
%  \begin{epigraph}[Tarde, 1919]{Olavo Bilac}
%  Ultima flor do Lacio, inculta e bela,\\
%  Es, a um tempo, esplendor e sepultura;\\
%  Ouro nativo, que, na ganga impura,\\
%  A bruta mina entre os cascalhos vela.
%  \end{epigraph}
\begin{epigraph}[NOTA]{AUTOR}
Se desejar, coloque  AQUI uma CITACAO
\end{epigraph}

% Resumo em Portugues
\resumo
COLOQUE O RESUMO. Se preferir, crie um arquivo separado e o inclua via comando "\"include.
% Palavras-chave do resumo em Portugues
\begin{keywords}
PALAVRAS-CHAVE.
\end{keywords}

% Resumo em Ingles
\abstract
COLOQUE O RESUMO EM INGLES. Se preferir, crie um arquivo separado e o inclua via comando "\"include.
% Palavras-chave do resumo em Ingles
\begin{keywords}
PALAVRAS-CHAVE EM INGLES.
\end{keywords}

% Sumario / Indice
% Comente para ocultar
\tableofcontents

% Lista de figuras
% Comente para ocultar
\listoffigures

% Lista de tabelas
% Comente para ocultar
\listoftables

%% Parte textual
\mainmatter

% Eh aconselhavel criar cada capitulo em um arquivo separado, digamos
% "capitulo1.tex", "capitulo2.tex", ... "capituloN.tex" e depois
% inclui-los com:
% \include{capitulo1}
% \include{capitulo2}
% ...
% \include{capituloN}
%
% Importante: Use \xchapter ao inves de \chapter, se quiser colocar texto antes do inicio do capitulo.

\xchapter{Introdu\c{c}\~{a}o}{Este eh o primeiro cap\'{\i}tulo, onde eu conto toda a historia deste trabalho, o problema, a solu\c{c}\~{a}o, etc.}

\lipsum

\xchapter{Revis\~{a}o Bibliogr\'{a}fica}{Neste cap\'{\i}tulo eu apresento todo o material que eu estudei durante a elabora\c{c}\~{a}o do trabalho.}

\lipsum

Livro \cite{demeyer2008} e  livro \cite{raymond1999}.

\chapter{Exemplos}

Figura \ref{default-regular} e tabela \ref{default-table}.
\begin{figure}[htbp]
\begin{center}
  \includegraphics[scale=0.5]{ufba.eps}[0.5]
\caption{Figura UFBA}
\label{default-regular}
\end{center}
\end{figure}

\begin{table}[htbp]
\caption{Tabela Exemplo}
\begin{center}
\begin{tabular}{|c|c|} 
\hline
elemento 11 & elemento 12 \\ \hline
elemento 21 & elemento 22 \\ \hline
elemento 31 & elemento 32 \\
\hline
\end{tabular}
\end{center}
\label{default-table}
\end{table}%

%% Parte pos-textual
\backmatter

% Apendices
% Comente se naoo houver apendices
\appendix

% Eh aconselhavel criar cada apendice em um arquivo separado, digamos
% "apendice1.tex", "apendice.tex", ... "apendiceM.tex" e depois
% inclui--los com:
% \include{apendice1}
% \include{apendice2}
% ...
% \include{apendiceM}

% Bibliografia
% É aconselhável utilizar o BibTeX a partir de um arquivo, digamos "biblio.bib".
% Para ajuda na criação do arquivo .bib e utilização do BibTeX, recorra ao
% BibTeXpress em www.cin.ufpe.br/~paguso/bibtexpress
\bibliographystyle{abntex2-alf}
\bibliography{biblio}

%% Fim do documento
\end{document}
%------------------------------------------------------------------------------------------%
