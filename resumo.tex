%A new powerful and flexible organization of documents can be obtained by mixing fuzzy and possibilistic clustering, in which documents can belong to more than one cluster simultaneously with different compatibility degrees with a particular topic. The topics are represented by clusters and the clusters are identified by one or more descriptors extracted by a proposed method. We aim to investigate whether the descriptors extracted after fuzzy and possibilistic clustering improves the flexible organization of documents. Experiments were carried using a collection of documents and we evaluated the descriptors ability to capture the essential information of the used collection. The results prove that the fuzzy possibilistic clusters descriptors extraction is effective and can improve the flexible organization of documents.

Diante da grande quantidade de informações geradas e armazenadas pela humanidade na atualidade, 
vários métodos foram propostos visando processar esses dados. Dentre esses dados, temos uma imensa
quantidade de dados textuais, que por sua vez são não estruturados. Com isso é notória a importância,
de organizar de maneira automatizada, esses documentos pelos assuntos ao qual se tratam. Em particular temos um conjunto de técnicas pertencentes ao campo de estudo da mineração de textos, que visam realizar a tarefa de extrair informações relevantes de documentos textuais. Esta tarefa de análise e extração de informações é 
comumente segmentada nas tarefas de coleta, pré-processamento dos documentos, agrupamento dos dados
e por fim a extração de descritores dos grupos obtidos na etapa de agrupamento. Os métodos de agrupamento podem ser separados então pela lógica matemática utilizada, que pode ser a lógica clássica ou a lógica fuzzy. Na lógica clássica, após o agrupamento, cada documento só poderá pertencer a um grupo, enquanto na lógica fuzzy, a pertinência do documento será distribuída entre os grupos. 
Se analisarmos a diversidade de conteúdo em documentos textuais, é trivial notar que frequentemente
um texto aborda um ou mais temas. Com isso é evidente a necessidade de desenvolver-se técnicas para
organizar de maneira flexível os documentos. Percebe-se então, que os métodos de agrupamento fuzzy,
se mostram coerentes com a realidade da estrutura dos documentos textuais.
% FALAR SOBRE OS MÈTODOS FUZZY UTILIZADOS, FCM, PFCM, PCM, HFCM, HPCM
% FALAR SOBRE A EXTRAÇÂO DE DESCRITORES
% FALAR SOBRE A MISTURA REALIZADA NA EXTRAÇÃO DE DESCRITORES
