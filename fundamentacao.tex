\section{Conjuntos e Lógica Fuzzy}
A lógica fuzzy é uma lógica multi valorada, onde os valores das variáveis pertencem ao intervalo
de [0,1], enquanto na lógica clássica os valores verdades só possuem os estados 0 ou 1 (também conhecido como valores {\it crisp\/}). Uma da mais importantes aplicações está no tratamento de precisão e incerteza. O que nos permite a modelar soluções mais adequadas para ambientes imprecisos e incertos.

Primeiramente introduzida em \cite{Zadeh1965}, onde o autor inicia a discussão definindo os
conjuntos fuzzy, sendo uma classe de objetos com valores contínuos de pertinência. Cada conjunto é então caracterizado por uma função de pertinência, a qual atribui a cada objeto do conjunto um grau de 
pertinência que varia entre zero e um. As operações matemáticas da teoria dos conjuntos, como
inclusão, união, intersecção, complemento, relação, etc., também são estendidas aos conjuntos fuzzy,
assim como várias propriedades dessas notações são definidas.

Uma das motivações da lógica fuzzy, vem da maneira como nosso cérebro classifica e rotula o mundo
real. Por exemplo, ao rotularmos uma pessoa como alta, estamos atribuindo ela ao grupo de pessoas 
altas. Assim como quando nos expressamos sobre o quanto um determinado dia está fazendo calor ou 
frio. O conjunto de pessoas altas ou dias frios, não se enquadra na sua totalidade na lógica 
clássica. Pois essa forma imprecisa de descrever o mundo a nossa volta, desempenha
um papel fundamental na forma de pensar humana, assim como também nas áreas de reconhecimento 
de padrões, comunicação e abstração\cite{Zadeh1965}.

\subsection{Definição de conjuntos fuzzy}

Seja X um espaço de objetos, com um elemento genérico {\it x\/}. 
Sendo $X= \big\{x\big\}$.

Um conjunto fuzzy $A$ em $X$ é caracterizado por uma função de pertinência $f_A(x)$, a qual associa
a cada elemento de $X$ um número real presente no intervalo de $[0,1]$, sendo o valor de $f_A(x)$
a representação do grau de pertinência de $x$ em $A$.

\subsection{Lógica fuzzy}

Antes da lógica fuzzy ser introduzida em \cite{Zadeh1965}, em 1930 Lukasiewics\cite{Chen2000} 
desenvolveu
a lógica n-valorada para $3 < n < \infty$, utilizando apenas os operadores lógicos de negação $-$ e implicação $\Rightarrow$. Dado então um inteiro positivo, $n > 3$, a lógica n-valorada assume 
valores verdade pertencente ao intervalo $[0,1]$, definidos pela seguinte partição igualmente 
espaçada: 
$$0 =  \frac{0}{n-1}, \frac{1}{n-1},\frac{2}{n-1},...,\frac{n-2}{n-1},\frac{n-1}{n-1} = 1$$
Para estender a lógica n-valorada para uma lógica com infinitos valores $2 \leq n \leq \infty$, 
\cite{Zadeh1965} modificou a lógica de Lukasiewics definindo os seguintes operadores lógicos:
$$\bar{a} = 1 -a$$
$$a \wedge b = min\{a,b\}$$
$$a \vee b = max\{a,b\}$$
$$a \Rightarrow b = min\{1, 1+b-a\}$$
$$a \Leftrightarrow  b = 1 - |a-b|$$

O objetivo da lógica fuzzy é prover mecanismos para tratar imprecisão e incerteza, se baseando na
teoria de conjuntos fuzzy e usando proposições imprecisas, de modo similar a lógica clássica 
usando proposições precisas baseadas na teoria dos conjuntos. 
Para entendermos essa noção, vejamos então um mesmo exemplo pela ótica do raciocínio clássico e 
em seguida usando as ferramentas para descrever imprecisão da lógica fuzzy. 
\begin{enumerate}[label=\alph*)]
  \item Todo texto com 100 palavras ou mais da área jurídica, tem como assunto o direito.
  \item O texto A com título as manifestações de junho, tem 100 palavras da área jurídica.
  \item O texto B com título política nas universidades, tem 99 palavras da área jurídica. 
  \item O texto A tem como assunto o direito e o texto B não tem como assunto o direito.
\end{enumerate}
Essa série de proposições ilustra o raciocínio empregado na lógica clássica, e seguindo as regras
de inferência conseguimos verificar que as sentenças estão corretas. No entanto é fácil notar que
a sentença d) não expressa muito bem o nosso entendimento sobre a temática dos textos.
Seria comum alguém substituir a sentença d), por e) O texto B fala um pouco sobre direito. 
Vamos então adicionar a imprecisão comum no mundo real as sentenças anteriores.
\begin{enumerate}[label=\alph*)]
  \item Todo texto que tem entre 50 e 100 palavras da área jurídica fala um pouco sobre direito. 
Enquanto todo texto que contenha 100 ou mais palavras da área jurídica fala bastante sobre direito.
  \item O texto A com título as manifestações de junho, tem 100 palavras da área jurídica.
  \item O texto B com título política nas universidades, tem 99 palavras da área jurídica. 
  \item O texto A fala bastante sobre direito, enquanto o texto B fala um pouco sobre direito.
\end{enumerate}
Esse tipo de dedução comumente utilizada no nosso dia a dia, não tem como ser tratada pela lógica
clássica. No entanto podemos lidar com esse tipo de inferência imprecisa, empregando a lógica fuzzy, a qual permite o uso de alguns termos linguísticos imprecisos como:
\begin{itemize}
  \item Predicados fuzzy: antigo, raro, caro, alto, rápido
  \item Quantificadores fuzzy: muito, pouco, quase, alguns
  \item Graus de verdade fuzzy: totalmente verdadeiro, verdadeiro, parcialmente falso, falso, definitivamente falso
\end{itemize}

\section{Organização Flexível de Documentos}

A organização de documentos tem sido uma área de fundamental importância no campo da
mineração de dados. Pois na era da informação, onde temos mídias sociais, internet das coisas
e computação móvel, produzindo diariamente uma elevada quantidade de dados estruturados e 
não estruturados. 

Se nos concentrarmos nos dados não estruturados, temos os dados textuais,
que por sua vez pertencem a uma ampla gama de tópicos que é constantemente atualizada, de maneira
que a organização deles em categorias não pode ser predefinida \cite{Carvalho2016}. 
Diante deste contexto, esse capítulo visa fundamentar as bases da organização flexível de documentos 
seguindo o modelo defendido em \cite{Nogueira2013},


\subsection{Pré-Processamento}
\subsection{Agrupamento Fuzzy}
\subsection{Extração de descritores}
