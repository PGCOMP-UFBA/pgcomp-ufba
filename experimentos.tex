\section{Considerações inicias}
A organização de uma coleção de documentos em vários tópicos, de modo que exista sobreposição
entre os grupos é um importante problema em sistemas de recuperação de informação(SRIs). Na
literatura diversas estratégias são utilizadas visando otimizar a organização flexível de
documentos, conforme foi abordado no capítulo anterior. Soma-se a isso o fato de que a maioria dos
métodos que adicionam flexibilidade ao processo, como por exemplo  os métodos de agrupamento, nem
sempre são desenvolvidos com o foco em documentos textuais. Que conforme foi abordado ao longo do
texto, possui características que acrescentam algumas dificuldades no processo, tais como a alta
dimensionalidade dos dados, assim como também usualmente são armazenados de maneira não estruturada.
E ainda com o crescente aumento do uso de tecnologias de produção de conteúdo, a quantidade de dados
textuais alcança grandes volumes de dados o que os enquadra no contexto do $Big Data$. 
Portanto esse cenário fortalece a importância de se conduzir pesquisas e investigações em torno da
organização flexível de documentos. Entretanto não é esperado que um método de agrupamento seja
totalmente adequado para todos os tipos de dados, incluindo os dados de alta dimensionalidade como
os dados textuais\cite{Steinbach2004}. Desta maneira esse capítulo tem como objetivo detalhar as 
contribuições
desta monografia a organização flexível de documentos, através da investigação dos impactos de se
utilizar a estratégia de se misturar o agrupamento fuzzy e possibilístico provida pelo algoritmo
PFCM. Onde este método de agrupamento pretende resolver os problemas dos elementos equidistantes e 
dos grupos
coincidentes, apresentados nas partições fuzzy e possibilística respectivamente. 

Conforme observa-se no capítulo 2, o algoritmo PFCM produz duas
partições, sendo um fuzzy e outra possibilística, o que induziu o presente trabalho a propor duas extensões do
método de extração de descritores Soft-wFDCL proposto por \cite{Nogueira2013}, que leva em
consideração apenas os valores de pertinências presentes na partição fuzzy. A primeira extensão
denominada Mixed-PDCL({\it Mixed - Possibilistic Fuzzy Descriptor Comes Last\/}), 
a qual contempla durante a extração de descritores as duas partições do PFCM.
E a segunda proposta é o método 
MixedW-PDCL({\it Mixed Weighted - Possibilistic Fuzzy Descriptor Comes Last\/}), 
que é uma extensão do Mixed-PDCL, porém ponderando as
contribuições das partições com base nos parâmetros $a$ e $b$ do PFCM. A última contribuição é a
proposta do método HPCM, que é uma extensão do método de agrupamento hierárquico HFCM, o qual
utiliza o algoritmo PCM no lugar do FCM para produzir a hierarquia.

Na primeira sessão deste capítulo é apresentado informações das bases de dados utilizadas, com as
suas características, origem e composição dos documentos. Nos capítulos seguintes é definido as
propostas sugeridas por essa monografia. E por fim os dados obtidos com os experimentos realizados.

\section{Informações das bases de dados}

Na mineração de dados e consequentemente nos trabalhos relacionados a organização flexível de
documentos, é comum se realizar a avaliação dos métodos propostos, conduzindo-se experimentos sobre
bases de dados existentes na literatura com essa finalidade\cite{Rossi2013}. Para isso, as bases
precisam estar apresentadas de maneira estruturada. Assim sendo, nesta pesquisa foi adotado o formato
{\it tf-idf\/}(\ref{eq:tfidf}) como forma de estruturar os dados presentes nas bases, 
de modo a capturar a
importância relativa dos termos nos documentos e na coleção. Cada coleção foi então disposta em dois
arquivos, sendo que o arquivo com extensão $.data$ contém $n$ linhas, onde cada linha constitui a 
representação de um
documento da coleção no formato {\it tf-idf\/}, para $n$ igual a quantidade de documentos 
presentes na
coleção, enquanto o que arquivo de extensão $.names$ possui a descrição dos $m$ termos existentes na
coleção dispostos um por linha.

Outro aspecto não menos importante, são as características particulares das coleções de dados. Pois
ressalta-se que para uma mais apurada análise dos resultados, é pertinente considerar as
particularidades de cada base, com a finalidade de encontrar possíveis justificativas para os
resultados apresentados, realizando-se indagações comparativas as peculiaridades sabidamente
conhecidas dos métodos analisados. Portanto o conjunto de características particulares de cada base
obtidos em \cite{Rossi2013} e adaptados a esta pesquisa, dar-se à como apresentado
na Tabela (\ref{table:datainfo}). 
\begin{table}[!htp]
  \centering
  \begin{tabular}{ |c|p{11cm}|}
    \hline
    {\bf documentos} & número de documentos presentes na coleção \\
    \hline
    {\bf termos} & número de termos existentes na coleção após o pré-processamento \\
    \hline
    {\bf \% zeros} & número relativo de zeros na {\it tf-idf\/}, ou seja quantifica o quanto a
matriz é esparsa \\
    \hline
    {\bf classes} & número de classes presentes na coleção \\
    \hline
    {\bf dp-class} & desvio padrão ao se considerar o percentual de documentos que
pertence a determinada classe na coleção \\
    \hline
    {\bf $>$classe} & percentual de documentos pertencentes a maior classe na coleção \\
    \hline
    {\bf n-gramas} & quantidade de termos considerados sequencialmente na coleção \\
    \hline
  \end{tabular}
  \caption{Importantes características presentes em coleções textuais}
  \label{table:datainfo}
\end{table}

Todos os experimentos conduzidos nesta pesquisa foram realizados com seis coleções de dados
textuais, e as características particulares de cada base está apresentada na Tabela
(\ref{table:datasets}).

A base Opinosis é composta de opiniões de consumidores a respeito das características de alguns
produtos, obtidas dos portais amazon.com, tripadvisor e edmunds.com. As opiniões presentes na base,
falam a respeito de serviços de hospedagem, dispositivos eletrônicos e carros. 

A coleção de documentos 20Newsgroup original contém aproximadamente 20000 documentos de notícias,
particionados em mais ou menos 20 temas. No entanto para os experimentos realizados nesta pesquisa, 
foi utilizado uma versão mais
compacta da coleção, contendo 2000 documentos pertencentes ao tema ciência, a qual contém os tópicos
sci.med, sci.space, sci.electronics e sci.med. Esta base tem-se mostrado bastante popular em
aplicações textuais de aprendizado de máquina\cite{Nogueira2015}, tais como agrupamento e 
classificação de textos.
Essa base foi coletada originalmente por Ken Lang para a pesquisa Newsweeder apresentada em
\cite{Lang1995}. 

\begin{table}[!htp]
  \centering
  \begin{tabular}{ |l|c c c c c c c|}
    \hline
    {\bf nome} & docs & termos & classes & \% zeros & dp-classe & $>$classe & n-gramas \\
    \hline
    {\bf Opinosis} & 51 & 842 & 3 & 95,73\% & 0,1 & 96,07\% & 1-grama \\
    \hline
    {\bf 20newsgroups} & 2000 & 11028 & 4 & 99,11\% & 0,1 & 25\% & 1-grama \\
    \hline
    {\bf Hitech} & 600 & 6925 & 6 & 97,93\% & 0,1 & 16,67\% & 1-grama \\
    \hline
    {\bf NSF} & 1600 & 2806 & 16 & 99,76\% & 0,1 & 3,12\% & 1-grama \\
    \hline
    {\bf WAP} & 1560 & 8070 & 20 & 98,51\% & 0,1 & 100\% & 1-grama \\
    \hline
    {\bf Reuters-21578} & 1052 & 3925 & 43 & 98,55\% & 0,1 & 8,55\% & 1-grama \\
    \hline
  \end{tabular}
  \caption{Características das coleções de documentos textuais}
  \label{table:datasets}
\end{table}




\section{Refinamento com os algoritmos PCM e PFCM}
\section{Método Mixed-PDCL}
\section{Método MixedW-PDCL}
\section{Método HPCM}
\section{Considerações finais}
