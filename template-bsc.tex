%% Template para dissertação/tese na classe UFBAthesis
%% versão 1.0
%% (c) 2005 Paulo G. S. Fonseca
%% (c) 2012 Antonio Terceiro
%% (c) 2014 Christina von Flach
%% www.dcc.ufba.br/~flach/ufbathesis

%% Carrega a classe ufbathesis
%% Opções: * Idiomas
%%           pt   - português (padrão)
%%           en   - inglês
%%         * Tipo do Texto
%%           bsc  - para monografias de graduação
%%           msc  - para dissertações de mestrado (padrão)
%%           qual - exame de qualificação de mestrado
%%           prop - exame de qualificação de doutorado
%%           phd  - para teses de doutorado
%%         * Um­dia
%%           scr  - para verão eletrônica (PDF) / consulte o guia do usuário
%%         * Estilo
%%           classic - estilo original a la TAOCP (deprecated)
%%           std     - novo estilo a la CUP (padrão)
%%         * Paginação
%%           oneside - para impressão em face única
%%           twoside - para impressão em frente e verso (padrão)
\documentclass[bsc, classic, a4paper]{ufbathesis}

%% Preambulo:
\usepackage[utf8]{inputenc}
\usepackage{graphicx}
\usepackage{lipsum}

% Universidade
\university{UNIVERSIDADE FEDERAL DA BAHIA}

% Endereço (cidade)
\address{Salvador}

% Instituto ou Centro Acadêmico
\institute{INSTITUTO DE MATEMÁTICA}

% Nome da biblioteca - usado na ficha catalográfica
\library{BIBLIOTECA REITOR MAC\^{E}DO COSTA}

% Programa de pós-graduação
\program{Programa de Graduação em Ciência da Computação}

% Area de titulação
\majorfield{CI\^{E}NCIA DA COMPUTA\c{C}\~{A}O}

% Titulo da dissertação
\title{Uma abordagem fuzzy híbrida para organização de documentos, utilizando os algoritmos de agrupamento possibilístico e fuzzy c means }

% Data da defesa
% e.g. \date{19 de fevereiro de 2003}
\date{1 de junho de 2016}

% Autor
% e.g. \author{Jose da Silva}
\author{Nilton Vasques Carvalho Junior}

% Orientador(a)
% Opção: [f] - para orientador do sexo feminino
% e.g. \adviser[f]{Profa. Dra. Maria Santos}
\adviser[f]{Profa. Dra. Tatiane Nogueira Rios}

% Orientador(a)
% Opção: [f] - para orientador do sexo feminino
% e.g. \coadviser{Prof. Dr. Pedro Pedreira}
% Comente se não se aplicar
%\coadviser{NOME DO(DA) CO-ORIENTADOR(A)}

%% Inicio do documento
\begin{document}

\pgcompfrontpage{}

%% Parte pre-textual
\frontmatter

\pgcomppresentationpage

% Ficha catalográfica
\authorcitationname{Carvalho, Nilton Vasques Jr.} % e.g. Terceiro, Antonio Soares de Azevedo
\advisercitationname{Rios, Tatiane Nogueira} % e.g. Chavez, Christina von Flach Garcia
\catalogtype{Monografia (Graduação)} % e.g. ``Tese (Doutorado)''
\catalogtopics{``1. Fuzzy C Means. 2. Organização de documents. 3. Lógica Fuzzy. 4. Mineração de dados.''} % e.g. ``1. Complexidade Estrutural. 2. Engenharia de Software''
\catalogcdd{NUMERO CDD} % e.g. ``CDD 20.ed. XXX.YY'' (esse número vai lhe ser dado pela biblioteca)
\catalogingsheet

% Termo de aprovação
\approvalsheet{Salvador, DIA de MES de ANO}{
   \comittemember{Profa. Dra. Tatiane Nogueira Rios}{Universidade Federal da Bahia}
   \comittemember{Prof. Dr. Professor 2}{Universidade 123}
   \comittemember{Profa. Dra. Professora 3}{Universidade ABC}
}

% Dedicatória
% Comente para ocultar
\begin{dedicatory}
Coloque sua DEDICATÓRIA AQUI.
\end{dedicatory}

% Agradecimentos
\acknowledgements
Coloque seus AGRADECIMENTOS AQUI.

% Epigrafe
%  \begin{epigraph}[Tarde, 1919]{Olavo Bilac}
%  Ultima flor do Lácio, inculta e bela,\\
%  Es, a um tempo, esplendor e sepultura;\\
%  Ouro nativo, que, na ganga impura,\\
%  A bruta mina entre os cascalhos vela.
%  \end{epigraph}
\begin{epigraph}[1687]{Isaac Newton}
  O que sabemos é uma gota, o que ignoramos é um oceano.
\end{epigraph}

% Resumo em Português
\resumo
%A new powerful and flexible organization of documents can be obtained by mixing fuzzy and possibilistic clustering, in which documents can belong to more than one cluster simultaneously with different compatibility degrees with a particular topic. The topics are represented by clusters and the clusters are identified by one or more descriptors extracted by a proposed method. We aim to investigate whether the descriptors extracted after fuzzy and possibilistic clustering improves the flexible organization of documents. Experiments were carried using a collection of documents and we evaluated the descriptors ability to capture the essential information of the used collection. The results prove that the fuzzy possibilistic clusters descriptors extraction is effective and can improve the flexible organization of documents.

Diante da grande quantidade de informações geradas e armazenadas pela humanidade na atualidade, 
vários métodos foram propostos visando processar esses dados. Dentre esses dados, temos uma imensa
quantidade de dados textuais, que por sua vez são não estruturados. Com isso é notória a importância,
de organizar de maneira automatizada, esses documentos pelos assuntos ao qual se tratam. Em particular temos um conjunto de técnicas pertencentes ao campo de estudo da mineração de textos, que visam realizar a tarefa de extrair informações relevantes de documentos textuais. Esta tarefa de análise e extração de informações é 
comumente segmentada nas tarefas de coleta, pré-processamento dos documentos, agrupamento dos dados
e por fim a extração de descritores dos grupos obtidos na etapa de agrupamento. Os métodos de agrupamento podem ser separados então pela lógica matemática utilizada, que pode ser a lógica clássica ou a lógica fuzzy. Na lógica clássica, após o agrupamento, cada documento só poderá pertencer a um grupo, enquanto na lógica fuzzy, a pertinência do documento será distribuída entre os grupos. 
Se analisarmos a diversidade de conteúdo em documentos textuais, é trivial notar que frequentemente
um texto aborda um ou mais temas. Com isso é evidente a necessidade de desenvolver-se técnicas para
organizar de maneira flexível os documentos. Percebe-se então, que os métodos de agrupamento fuzzy,
se mostram coerentes com a realidade multi temática dos documentos textuais. Por sua vez, o método FCM(fuzzy c means), que é uma adaptação do clássico k means, se propõe a identificar e separar uma coleção de documentos em grupos, respeitando a lógica multi valorada, permitindo assim que um documento pertença a um ou mais grupos. No entanto o FCM possui algumas falhas conhecidas, o que motivou a pesquisa e desenvolvimento de métodos alternativos e baseados no FCM, com o propósito de sanar estes problemas. Este é o caso dos métodos PCM(Possibilístico C Means) e PFCM(Possibilístico C Means). 
Para então avaliarmos corretamente o resultado do agrupamento e a qualidade da organização flexível 
de documentos, é preciso extrair corretamente os descritores dos grupos obtidos, levando em 
consideração a relevância de determinado termo para cada grupo. Com isso temos um cenário no qual é 
preciso combinar métodos de agrupamento fuzzy com métodos de extração de descritores, para obtermos
uma bom resultado no processo de organização dos documentos. A investigação e refinamento dessa 
combinação de métodos, foi a motivação do presente trabalho. Como resultado desse trabalho foi,
proposto extender os experimentos referentes a organização flexível de documentos, utilizando
novos métodos de agrupamento fuzzy existentes na literatura, como o PCM e o PFCM. Assim como também 
foi proposto os métodos de extração de descritores: i) Mixed-PFDCL 
({ \it Mixed - Possibilistic Fuzzy Descriptor Comes Last\/ }), que se utiliza da abordagem híbrida do 
algoritmo PFCM, misturando assim descritores fuzzy e possibilísticos. ii) MixedW-PFDCL
({ \it Mixed Weighted - Possibilistic Fuzzy Descriptor Comes Last\/ }), 
onde além de misturar descritores 
fuzzy e possibilístico, leva em consideração os parâmetros de ponderação do método PFCM.
Além dos métodos de extração de descritores, foi conduzido um estudo dos impactos de se utilizar o 
algoritmo PCM, no método de agrupamento hierárquico HFCM, o que resultou no método de 
agrupamento hierárquico HPCM ({ \it Hierarchical Possibilistic C Means\/ }).
% FALAR SOBRE OS MÈTODOS FUZZY UTILIZADOS, FCM, PFCM, PCM, HFCM, HPCM
% FALAR SOBRE A EXTRAÇÂO DE DESCRITORES
% FALAR SOBRE A MISTURA REALIZADA NA EXTRAÇÃO DE DESCRITORES

% Palavras-chave do resumo em Português
\begin{keywords}
agrupamento fuzzy, agrupamento possibilístico, organização flexível de documentos, 
mineração de textos
\end{keywords}

% Resumo em Ingles
\abstract
A new powerful and flexible organization of documents can be obtained by mixing fuzzy and possibilistic clustering, in which documents can belong to more than one cluster simultaneously with different compatibility degrees with a particular topic. The topics are represented by clusters and the clusters are identified by one or more descriptors extracted by a proposed method. We aim to investigate whether the descriptors extracted after fuzzy and possibilistic clustering improves the flexible organization of documents. Experiments were carried using a collection of documents and we evaluated the descriptors ability to capture the essential information of the used collection. The results prove that the fuzzy possibilistic clusters descriptors extraction is effective and can improve the flexible organization of documents.

% Palavras-chave do resumo em Ingles
\begin{keywords}
fuzzy clustering, possibilistic clustering, flexible organization, documents, text mining
\end{keywords}

% Sumario / Índice
% Comente para ocultar
\tableofcontents

% Lista de figuras
% Comente para ocultar
\listoffigures

% Lista de tabelas
% Comente para ocultar
\listoftables

%% Parte textual
\mainmatter

% Eh aconselhável criar cada capitulo em um arquivo separado, digamos
% "capitulo1.tex", "capitulo2.tex", ... "capituloN.tex" e depois
% inclui-los com:
% \include{capitulo1}
% \include{capitulo2}
% ...
% \include{capituloN}
%
% Importante: Use \xchapter ao invés de \chapter, se quiser colocar texto antes do inicio do capitulo.

\xchapter{Introdução}{Uma breve introdução sobre do que se trata esta monografia e a maneira como o texto está organizado.}

%A new powerful and flexible organization of documents can be obtained by mixing fuzzy and possibilistic clustering, in which documents can belong to more than one cluster simultaneously with different compatibility degrees with a particular topic. The topics are represented by clusters and the clusters are identified by one or more descriptors extracted by a proposed method. We aim to investigate whether the descriptors extracted after fuzzy and possibilistic clustering improves the flexible organization of documents. Experiments were carried using a collection of documents and we evaluated the descriptors ability to capture the essential information of the used collection. The results prove that the fuzzy possibilistic clusters descriptors extraction is effective and can improve the flexible organization of documents.

Diante da grande quantidade de informações geradas e armazenadas pela humanidade na atualidade, 
vários métodos foram propostos visando processar esses dados. Dentre esses dados, temos uma imensa
quantidade de dados textuais, que por sua vez são não estruturados. Com isso é notória a importância,
de organizar de maneira automatizada, esses documentos pelos assuntos ao qual se tratam. Em particular temos um conjunto de técnicas pertencentes ao campo de estudo da mineração de textos, que visam realizar a tarefa de extrair informações relevantes de documentos textuais. Esta tarefa de análise e extração de informações é 
comumente segmentada nas tarefas de coleta, pré-processamento dos documentos, agrupamento dos dados
e por fim a extração de descritores dos grupos obtidos na etapa de agrupamento. Os métodos de agrupamento podem ser separados então pela lógica matemática utilizada, que pode ser a lógica clássica ou a lógica fuzzy. Na lógica clássica, após o agrupamento, cada documento só poderá pertencer a um grupo, enquanto na lógica fuzzy, a pertinência do documento será distribuída entre os grupos. 
Se analisarmos a diversidade de conteúdo em documentos textuais, é trivial notar que frequentemente
um texto aborda um ou mais temas. Com isso é evidente a necessidade de desenvolver-se técnicas para
organizar de maneira flexível os documentos. Percebe-se então, que os métodos de agrupamento fuzzy,
se mostram coerentes com a realidade multi temática dos documentos textuais. Por sua vez, o método FCM(fuzzy c means), que é uma adaptação do clássico k means, se propõe a identificar e separar uma coleção de documentos em grupos, respeitando a lógica multi valorada, permitindo assim que um documento pertença a um ou mais grupos. No entanto o FCM possui algumas falhas conhecidas, o que motivou a pesquisa e desenvolvimento de métodos alternativos e baseados no FCM, com o propósito de sanar estes problemas. Este é o caso dos métodos PCM(Possibilístico C Means) e PFCM(Possibilístico C Means). 
Para então avaliarmos corretamente o resultado do agrupamento e a qualidade da organização flexível 
de documentos, é preciso extrair corretamente os descritores dos grupos obtidos, levando em 
consideração a relevância de determinado termo para cada grupo. Com isso temos um cenário no qual é 
preciso combinar métodos de agrupamento fuzzy com métodos de extração de descritores, para obtermos
uma bom resultado no processo de organização dos documentos. A investigação e refinamento dessa 
combinação de métodos, foi a motivação do presente trabalho. Como resultado desse trabalho foi,
proposto extender os experimentos referentes a organização flexível de documentos, utilizando
novos métodos de agrupamento fuzzy existentes na literatura, como o PCM e o PFCM. Assim como também 
foi proposto os métodos de extração de descritores: i) Mixed-PFDCL 
({ \it Mixed - Possibilistic Fuzzy Descriptor Comes Last\/ }), que se utiliza da abordagem híbrida do 
algoritmo PFCM, misturando assim descritores fuzzy e possibilísticos. ii) MixedW-PFDCL
({ \it Mixed Weight - Possibilistic Fuzzy Descriptor Comes Last\/ }), 
onde além de misturar descritores 
fuzzy e possibilístico, leva em consideração os parâmetros de ponderação do método PFCM.
Além dos métodos de extração de descritores, foi conduzido um estudo dos impactos de se utilizar o 
algoritmo PCM, no método de agrupamento hierárquico HFCM, o que resultou no método de 
agrupamento hierárquico HPCM ({ \it Hierarchical Possibilistic C Means\/ }).
% FALAR SOBRE OS MÈTODOS FUZZY UTILIZADOS, FCM, PFCM, PCM, HFCM, HPCM
% FALAR SOBRE A EXTRAÇÂO DE DESCRITORES
% FALAR SOBRE A MISTURA REALIZADA NA EXTRAÇÃO DE DESCRITORES


\xchapter{Fundamentação Teórica}{Este capítulo tem como objetivo fundamentar as bases necessárias dos campos de estudos utilizados nesta monografia.}
\section{Lógica Fuzzy}
\section{Pré-Processamento}
\section{Agrupamento Fuzzy}
\section{Extração de descritores}


\xchapter{Revisão Bibliográfica}{Revisão de todo material utilizado desde a fase de pesquisa e implementação até a execução dos experimentos.}

% Strings de busca 
%
% (((((clustering) OR cluster labeling) OR cluster description) AND fuzzy AND( document OR text
% mining)))
% http://ieeexplore.ieee.org/search/searchresult.jsp?queryText=(((((clustering)%20OR%20cluster%20labeling)%20OR%20cluster%20description)%20AND%20fuzzy%20%20AND(%20document%20OR%20text%20mining)))&ranges=2010_2016_Year&matchBoolean=true&searchField=Search_All

% ((clustering OR "cluster label*" OR "cluster descriptors") AND fuzzy AND (document OR "text
% mining" OR "document organization" OR "soft document" OR "text data"))
% http://ieeexplore.ieee.org/search/searchresult.jsp?queryText=((clustering%20OR%20.QT.cluster%20label*.QT.%20OR%20.QT.cluster%20descriptors.QT.)%20AND%20fuzzy%20AND%20(document%20OR%20.QT.text%20mining.QT.%20OR%20.QT.document%20organization.QT.%20OR%20.QT.soft%20document.QT.%20OR%20.QT.text%20data.QT.))&sortType=desc_p_Publication_Year&matchBoolean=true&searchField=Search_All

\section{Considerações Iniciais}

A proposta de organização flexível de documentos está relacionada a vários campos de estudo, como
ficou evidenciado na fundamentação teórica. Por isso a literatura existente para essa proposta é
bastante rica e densa. Portanto com o propósito de otimizar a atividade de pesquisa e seleção do
conhecimento científico produzido a respeito do tema, foram utilizados algumas técnicas de revisão
sistemática de literatura ($SLR\ -\ Sistematic\ Literature\ Review$) utilizadas em \cite{Rios2010}.
Com o objetivo de estabelecer critérios mais precisos na fase inicial da descoberta de conteúdo
científico relacionado ao tema. Foi então adotada uma técnica comum ao método SLR, que consiste na
elaboração de uma uma string de busca, usando operadores lógicos. Estabelecendo assim uma maneira
mais objetiva para a obtenção de resultados relevantes a proposta dessa monografia.  Portanto,
levando em consideração os tópicos chaves e a proposta desse trabalho, foi construída a seguinte
string de busca: 

\begin{multline} (clustering\ OR\ "cluster\ label*"\ OR\ "cluster\ descriptors")\ AND\ fuzzy \\ AND\
  (document\ OR\ "text\ mining"\ OR\ "document\ organization"\ OR\ \\ "soft\ document"\ OR\ "text\
data") \label{eq:busca} \end{multline}

Devido o amplo acervo de publicações científicas presentes no repositório
IEEExplore\footnote{http://ieeexplore.ieee.org/}, assim como também a possibilidade de se utilizar
operadores lógicos e buscas parametrizadas. Foi realizado então uma busca no repositório IEEExplore,
restringindo o período de resultados entre os anos de 2010 e 2016, permitindo então que os
resultados obtidos fossem mais recentes.

Com base nos resultados obtidos, foi realizada a leitura dos títulos e resumos dos artigos, com o
propósito de descartar resultados com baixa relevância para essa pesquisa.  Durante a fase de
leitura parcial dos resultados da busca, foram agrupados os artigos em três categorias: agrupamento
fuzzy, extração de descritores e organização flexível de documentos.  As publicações selecionadas e
direcionadas para a categoria de agrupamento fuzzy, foram as que possuíam propostas de alteração de
métodos de agrupamento existentes ou novos métodos. Enquanto artigos que tinham como conteúdo a
análise dos termos de uma coleção, critério de seleção de termos ou atribuição de termos a grupos de
documentos, foram agrupados na categoria de extração de descritores. Por fim, artigos mais gerais,
propondo métodos ou realizando revisões de métodos, pertinentes ao processo de organização de
documentos textuais, foram categorizados no grupo de organização flexível de documentos.

Para complementar os resultados obtidos foram adicionados artigos de alta relevância para o tema, e
que apesar de serem antigos, ainda são amplamente citados em pesquisas recentes. Muitos desses
artigos como é o caso do método FCM proposto em \cite{Bezdek1984}, são pilares fundamentais para o
tema.

Nas próximas seções contém a revisão das pesquisas selecionadas, onde é elucidado os pontos chaves
de cada pesquisa, a definição das propostas contida nos artigos e por fim a conexão com o objetivo
dessa monografia.

\section{Organização Flexível de Documentos}

Após a proposição da lógica fuzzy que se propunha a lidar com a incerteza e imprecisão em
\cite{Zadeh1965}, foi possível a elaboração de diversos métodos que se utilizassem dos benefícios da
lógica fuzzy e aplicassem a diversos problemas do mundo real. Este é o caso da organização de
documentos, que por não ser uma tarefa precisa, necessita de uma certa flexibilidade no processo.

\cite{Matsumoto10} informa que os mecanismos adotados em sistemas de recuperação de informação
(SRI), tais como buscadores web, estão dispostos em dois grupos. Sendo que o primeiro tem como foco
o usuário realizando a busca, a qual é comumente chamada de busca web personalizada.  Nessa
abordagem os resultados obtidos são ordenados de acordo com a relevância do resultado para o
usuário. Para calcular essa relevância, os buscadores realizam tarefas de coleta de dados dos
usuários e comparação das preferências com demais usuários do sistema.  Enquanto na segunda
abordagem os resultados da busca é categorizado em grupos, permitindo assim que o usuário decida em
qual grupo ele pretende visualizar as informações. Por exemplo, quando um usuário pesquisar pelo
termo java, os resultados poderiam ser agrupados nas seções: máquina virtual, linguagem java,
programas em java, oracle e etc. Seguindo essa linha de categorização de resultados em SRIs,
\cite{MarcaciniR10} propõe uma abordagem de agrupamento incremental e hierárquico para construção
dos tópicos dos documentos, a qual permite a atualização das categorias a medida que novos
documentos são adicionados sem realizar a etapa de agrupamento novamente. É possível a visualização
dessa abordagem de categorização hierárquica, através da ferramenta online
Torch\footnote{http://sites.labic.icmc.usp.br/torch/webcluster/}, dos autores do artigo.

Com o surgimento de várias tecnologias, como mídias sociais, computação ubíqua, internet das coisas 
e principalmente os dispositivos móveis, que ultrapassou os 7 bilhões de
dispositivos\footnote{Segundo o relatório do The Mobile Economy disponível em
\url{http://www.gsmamobileeconomy.com/GSMA_Global_Mobile_Economy_Report_2015.pdf}, a quantidade de
dispositivos móveis (smartphones e tablets) atingiu o total de 7,517 bilhões no ano de 2015.} no ano
de 2015. Onde por sua vez, todas essas tecnologias produzem uma abundante quantidade de dados não
estruturados, dificultando a tarefa de métodos de mineração de dados, e por consequência
também os métodos de organização de documentos já existentes. A esse cenário é usualmente atribuído
o nome de $Big\ Data$. Sendo assim novas pesquisas como \cite{Havens2012} e \cite{Kumar2015} tem 
sido conduzidas, focadas em bases com imensas quantidades de dados. Segundo \cite{Havens2012} 
existem duas abordagens principais para otimizar o agrupamento de dados que se encaixam na categoria
$Very\ Large$ (Tabela \ref{table:datasize}, a primeira consiste na técnica de agrupamento
distribuído incremental e o agrupamento por amostragem progressiva ou aleatória. Nos métodos que
usam a técnica de amostragem, primeiramente é selecionado uma amostra com os dados representativos 
da coleção, depois é realizado o agrupamento, e em seguida é generalizado o agrupamento para o
restante dos dados. Um dos métodos mais populares baseado em amostragem é o algoritmo 
$generalized\ extensible\ fast\ FCM$ (geFFCM)\cite{Havens2012}. O geFFCM utiliza amostragem
progressiva para se obter uma versão reduzida dos dados, de maneira que a mesma preserve as
características da base original. Porém segundo \cite{Havens2012}, a técnica de amostragem do geFFCM
é ineficiente para dados na categoria $Very\ Large$, o que levou os autores a propor uma extensão do
geFFCM com uma melhoria na forma de realizar a amostragem dos dados, utilizando uma metodologia de
seleção aleatória.

\begin{table}[!htp]
  \centering
  \begin{tabular}{ |c|c c c c c|}
    \hline
    Bytes & $10^6$ & $10^8$ & $10^{10}$ & $10^{12}$ & $10^{>12}$ \\
    \hline
    "tamanho" & medium & large & huge & monster & very large \\
    \hline
  \end{tabular}
  \label{table:datasize}
  \caption{Classificação das bases de dados de acordo com o seu tamanho\cite{Havens2012}}
\end{table}

De acordo com \cite{Deng2010}, a organização flexível de dados através do algoritmo FCM possui
uma falta de estabilidade, pois como a inicialização do FCM depende da aleatoriedade, o resultado
final do agrupamento pode variar a cada inicialização. Assim como os dados presentes em bases de
dados textuais são de alta dimensionalidade.  Os autores propuseram então um modelo de inicialização
da partição que extrai da coleção medidas de peso, raio e objetos mais representativos para orientar
a inicialização da partição inicial. A respeito do problema da dimensionalidade, \cite{Deng2010}
sugere a redução da matrix documentos x termos, usando uma medida estatística para avaliar a
qualidade dos termos presentes na coleção, descartando assim os termos considerados de baixa
qualidade e consequentemente reduzindo a largura da matriz.

\cite{Karami2015} propõe um modelo para análise textual de documentos médicos. Um dos pontos
interessantes propostos pelo autor é a utilização do agrupamento fuzzy na etapa de
pré-processamento e ponderação dos termos, antes de realizar o agrupamento e classificação. 
O agrupamento fuzzy é aplicado a coleção de termos presentes na coleção, e ao contrário do
agrupamento na etapa pós processamento, a pertinência ocorre da palavra a um tópico ou grupo, 
de maneira que termos com alta pertinência possuem significados
semânticos mais próximos. Essa aproximação semântica é realizada com base em um vocabulário
predefinido.
 


Livro \cite{demeyer2008} e  livro \cite{raymond1999}.

\chapter{Exemplos}

Figura \ref{default-regular} e tabela \ref{default-table}.
\begin{figure}[htbp]
\begin{center}
  \includegraphics[scale=0.5]{ufba.eps}[0.5]
\caption{Figura UFBA}
\label{default-regular}
\end{center}
\end{figure}

\begin{table}[htbp]
\caption{Tabela Exemplo}
\begin{center}
\begin{tabular}{|c|c|} 
\hline
elemento 11 & elemento 12 \\ \hline
elemento 21 & elemento 22 \\ \hline
elemento 31 & elemento 32 \\
\hline
\end{tabular}
\end{center}
\label{default-table}
\end{table}%

%% Parte pos-textual
\backmatter

% Apendices
% Comente se naoo houver apendices
\appendix

% Eh aconselhavel criar cada apendice em um arquivo separado, digamos
% "apendice1.tex", "apendice.tex", ... "apendiceM.tex" e depois
% inclui--los com:
% \include{apendice1}
% \include{apendice2}
% ...
% \include{apendiceM}

% Bibliografia
% É aconselhável utilizar o BibTeX a partir de um arquivo, digamos "biblio.bib".
% Para ajuda na criação do arquivo .bib e utilização do BibTeX, recorra ao
% BibTeXpress em www.cin.ufpe.br/~paguso/bibtexpress
\bibliographystyle{abntex2-alf}
\bibliography{biblio}

%% Fim do documento
\end{document}
%------------------------------------------------------------------------------------------%
